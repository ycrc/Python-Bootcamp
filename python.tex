\documentclass[10pt]{beamer}
\usetheme{Singapore}
\usecolortheme{default}
\usecolortheme{orchid}
\useoutertheme{infolines}
\useinnertheme[shadow=true]{rounded}

\newcommand\smallfont{\fontsize{8pt}{7.2}\selectfont}

\usepackage{multimedia}

\title{Introduction to Python}
\titlegraphic{\includegraphics[height=2.0cm]{../logo.png}}
\logo{\includegraphics[height=0.5cm]{../logo.png}}
\author{{Stephen Weston} \and {Robert Bjornson}}
\institute[Yale]{
  Yale Center for Research Computing \\
  Yale University
}
\date{Dec 2015}
\begin{document}


%----------- titlepage ----------------------------------------------%
\begin{frame}[plain]
  \titlepage
\end{frame}

%----------- slide --------------------------------------------------%
\begin{frame}
\frametitle{To get help}
\begin{itemize}
\item Send an email to: \url{hpc@yale.edu}
\item Read documentation at: \url{http://research.computing.yale.edu/hpc-support}
\item Email us directly:
\begin{itemize}
\item Stephen.weston@yale.edu, Office hours at CSSSI on Wednesday morning from 9 to 12 or by appointment
\item Robert.bjornson@yale.edu, Office hour by appointment
\end{itemize}
\end{itemize}
\end{frame}

%----------- slide --------------------------------------------------%
\begin{frame}
\frametitle{What is the Yale Center for Research Computing?}

\begin{itemize}
\item Newly created Center to support your research computing needs.
\item Applications specialists and system engineers.
\item Available to consult with and educate users
\item Manage compute clusters
\end{itemize}
\end{frame}

%----------- slide --------------------------------------------------%
\begin{frame}
\frametitle{Why Python?}

\begin{itemize}
\item Free, portable, easy to learn
\item Intuitive, easy to learn
\item Ideal for rapid prototyping but also for large applications
\item Reasonably efficient
\item Numerous extensions (modules)
\end{itemize}
\end{frame}

%----------- slide --------------------------------------------------%
\begin{frame}
\frametitle{You can use Python to...}
\begin{itemize}
\item Convert or filter files
\item Automate repetitive tasks
\item Compute statistics
\item Build processing pipelines
\item Simple web applications
\item ...
\end{itemize}
\end{frame}

%----------- slide --------------------------------------------------%
\begin{frame}[fragile]
\frametitle{Basic Python Types and Assignment}

\begin{verbatim}
radius=2
pi=3.14
diam=radius*2
area=pi*(radius**2)
title="fun with strings"
pi="cherry"
pi=3.1415926535897932384626433832795028841971693993751058\
2097494459230781640628620899862803482534211706798214808651\
32823066470938446095505822317253594081284811174502841027019\
385211055596446229489549303819644288109756659334461284756482
delicious=True
\end{verbatim}

Note:
\begin{itemize}
\item variables do not need to be declared or typed
\item integers and floats can be used together
\item the same variable can hold different types
\item lines can be broken using \
\item python supports arbitrary length floating point numbers
\end{itemize}
\end{frame}

%----------- slide --------------------------------------------------%
\begin{frame}[fragile]
\frametitle{Other Python Types: lists}

Lists are like arrays in other languages.  
\begin{verbatim}
>>> l=[1,2,3,4,5,6,7,8,9,10]
>>> l[5]
6
>>> l[3:5]
[4, 5]
>>> l[5:]
[6, 7, 8, 9, 10]
>>> l[5:-3]
[6, 7]
>>> l[2]=3.145
>>> l[3]="pi"
>>> l
[1, 2, 3.145, 'pi', 5, 6, 7, 8, 9, 10]
>>> len(l)
10
\end{verbatim}

\end{frame}

%----------- slide --------------------------------------------------%
\begin{frame}[fragile]
\frametitle{Other Python Types: strings}

Strings are fully featured types in python.

\begin{verbatim}
>>> s="Donald"
>>> s[0:3]
'Don'
>>> s+" Duck"
'Donald Duck'
>>> s[0]="R"
Traceback (most recent call last):
  File "<stdin>", line 1, in <module>
TypeError: 's' object does not support item assignment
>>> len(s)
6
>>> s.upper()
'DONALD'
\end{verbatim}

Note:
\begin{itemize}
\item strings cannot be modified
\item strings can be concatenated and sliced much like lists
\item strings are objects with lots of useful methods
\end{itemize}

\end{frame}


%----------- slide --------------------------------------------------%
\begin{frame}[fragile]
\frametitle{Other Python Types: dictionaries}

Dicts are like hash tables in other languages.  
\begin{verbatim}
>>> coins={'penny':1, 'nickle':5, 'dime':10, 'quarter':25}
>>> coins['penny']
1
>>> coins.keys()
['quarter', 'nickle', 'penny', 'dime']
>>> coins['half']=50
>>> coins
{'quarter': 25, 'nickle': 5, 'penny': 1, 'half': 50, 'dime': 10}
>>> len(coins)
5
\end{verbatim}

Note:
\begin{itemize}
\item dicts associate keys with values, which can be of (almost) any type 
\item dicts have length, but are not ordered
\item looking up values in dicts is very fast, even if the dict is BIG.
\end{itemize}

\end{frame}

%----------- slide --------------------------------------------------%
\begin{frame}[fragile]
\frametitle{Control Flow Statements: if}

If-then statements allow you to do a test, and then do something
based on the result:

\begin{verbatim}
>>> import random
>>> v=random.randint(0,100)
>>> if v < 50:
...   print 'got a little one', v
... else:
...   print 'got a big one', v
... 
got a big one 93
\end{verbatim}

Note that the else clause is optional.

\end{frame}

%----------- slide --------------------------------------------------%
\begin{frame}[fragile]
\frametitle{Control Flow Statements: while}

While statements execute one or more statements repeatedly until the 
test is false:

\begin{verbatim}
>>> import random
>>> count=0
>>> while count<100:
...   count+=random.randint(0,10)
...   print str(count)+" ",
... 
5  11  19  19  21  28  35  37  47  53  53  57  58  59  60  66  71  
75  82  86  94  101 
\end{verbatim}

\end{frame}


%----------- slide --------------------------------------------------%
\begin{frame}[fragile]
\frametitle{Control Flow Statements: for}

For statements take some sort of iterable object and loop once for 
every value.

\begin{verbatim}
>>> for fruit in ['apple', 'orange', 'banana']:
...   print fruit, 
... 
apple orange banana
>>> for i in range(5):
...   print i,
... 
0 1 2 3 4 

\end{verbatim}

\end{frame}

%----------- slide --------------------------------------------------%
\begin{frame}[fragile]
\frametitle{Control Flow Statements: altering loops}

While and For loops can skip steps (continue) or terminate early (break).

\begin{verbatim}
>>> for i in range(10):
...   if i%2: continue
...   print i, 
... 
0 2 4 6 8
>>> for i in range(10):
...   if i>5: break
...   print i,
... 
0 1 2 3 4 5
\end{verbatim}

\end{frame}

%----------- slide --------------------------------------------------%
\begin{frame}[fragile]
\frametitle{Note on blocks of code}

In the previous example:
\begin{verbatim}
>>> for i in range(10):
...   if i>5: break
...   print i,
\end{verbatim}

How did we know that \textit{print i} was part of the loop?

Many programming languages use \{ \} or Begin End to delineate blocks of
code to treat as a single unit.

Python uses white space (blanks).  To define a block of code, indent the block.
By convention and for readability, indent 3 spaces.

Many editors will do this for you.

\end{frame}

%----------- slide --------------------------------------------------%
\begin{frame}[fragile]
\frametitle{Functions}

TODO

\end{frame}

%----------- slide --------------------------------------------------%
\begin{frame}[fragile]
\frametitle{File Formatter example}

Task: given a file of hundreds or thousands of lines:

\begin{verbatim}
FCID,Lane,Sample_ID,SampleRef,index,Description,Control,Recipe,...
160212,1,A1,human,TAAGGCGA-TAGATCGC,None,N,Eland-rna,Mei,Jon_mix10
160212,1,A2,human,CGTACTAG-CTCTCTAT,None,N,Eland-rna,Mei,Jon_mix10
160212,1,A3,human,AGGCAGAA-TATCCTCT,None,N,Eland-rna,Mei,Jon_mix10
160212,1,A4,human,TCCTGAGC-AGAGTAGA,None,N,Eland-rna,Mei,Jon_mix10
...
\end{verbatim}

Remove the last 3 letters from the 5th column:

\begin{verbatim}
FCID,Lane,Sample_ID,SampleRef,index,Description,Control,Recipe,...
160212,1,A1,human,TAAGGCGA-TAGAT,None,N,Eland-rna,Mei,Jon_mix10
160212,1,A2,human,CGTACTAG-CTCTC,None,N,Eland-rna,Mei,Jon_mix10
160212,1,A3,human,AGGCAGAA-TATCC,None,N,Eland-rna,Mei,Jon_mix10
160212,1,A4,human,TCCTGAGC-AGAGT,None,N,Eland-rna,Mei,Jon_mix10
...
\end{verbatim}
\end{frame}

%----------- slide --------------------------------------------------%
\begin{frame}[fragile]
\frametitle{File Formatter example (cont)}
\smallfont
In pseudocode we might write:
\begin{verbatim}
 open the input file
 read the first header line, and print it out
 for each remaining line in the file
   read the line
   find the value in the 5th column
   truncate it by removing the last three letters
   put the line back together
   print it out
\end{verbatim}
In Python:
\begin{verbatim}
import sys
fp=open(sys.argv[1])
hdr=fp.readline().strip()
print hdr
for l in fp:
    flds=l.strip().split(',')
    flds[4]=flds[4][:-3]
    print ','.join(flds)
\end{verbatim}
\end{frame}

%----------- slide --------------------------------------------------%
\begin{frame}[fragile]
\frametitle{File Formatter example (cont)}

\textit{open the input file}

\begin{verbatim}
import sys
fp=open(sys.argv[1])
\end{verbatim}

Sys is a system module with a number of useful methods.  

sys.argv() returns the command line as an array of strings.  Argv[0] is the command, argv[1] is the first
argument, etc.

Open takes a filename, and returns a ``file pointer''.  We'll use that
to read from the file.

\end{frame}

%----------- slide --------------------------------------------------%
\begin{frame}[fragile]
\frametitle{File Formatter example (cont)}

\textit{read the first header line, and print it out}

\begin{verbatim}
hdr=fp.readline().strip()
print hdr
\end{verbatim}

We'll call readline() on the file pointer to get a single line from the file.
(the header line).  Strip() removes the return at the end of the line.
Then we print it.

\end{frame}

%----------- slide --------------------------------------------------%
\begin{frame}[fragile]
\frametitle{File Formatter example (cont)}

\textit{for each remaining line in the file}

\textit{read the line}


\begin{verbatim}
for l in fp:
  ...
\end{verbatim}

A file pointer is a special object called an iterator.  Instead of 
explicitly calling readline() for each line, we can just loop on the file 
pointer, getting one line each time. Since we already read the header, we 
won't get that line.

\end{frame}

%----------- slide --------------------------------------------------%
\begin{frame}[fragile]
\frametitle{File Formatter example (cont)}

\textit{find the value in the 5th column}

\textit{truncate it by removing the last three letters}

\begin{verbatim}
    flds=l.strip().split(',')
    flds[4]=flds[4][:-3]
\end{verbatim}

Just like before, we strip the return from the line.  We split it into
individual elements where we find commas.  The 5th field is referenced by
flds[4], since python starts indexing with 0.  [:-3] takes all characters
of the string until the last 3.
  
\end{frame}

%----------- slide --------------------------------------------------%
\begin{frame}[fragile]
\frametitle{File Formatter example (cont)}


\textit{put the line back together}

\textit{print it out}

\begin{verbatim}
    print ','.join(flds)
\end{verbatim}

Join takes a list of strings, and combines them into one string using the 
string provided. Then we just print that string.  
 
\end{frame}

%----------- slide --------------------------------------------------%
\begin{frame}[fragile]
\frametitle{File Formatter example (cont)}

Reviewing:

\begin{verbatim}
import sys
fp=open(sys.argv[1])
hdr=fp.readline().strip()
print hdr
for l in fp:
    flds=l.strip().split(',')
    flds[4]=flds[4][:-3]
    print ','.join(flds)
\end{verbatim}


We would invoke it like this:
\begin{verbatim}
$ python fixfile.py badfile.txt > fixedfile.txt
\end{verbatim}

\end{frame}

%----------- slide --------------------------------------------------%
\begin{frame}[fragile]
\frametitle{Some variations on the theme}

We could skip certain lines

\begin{verbatim}
import sys
fp=open(sys.argv[1])
hdr=fp.readline().strip()
print hdr
for l in fp:
    if l[3] != 'human': continue
    flds=l.strip().split(',')
    flds[4]=flds[4][:-3]
    print ','.join(flds)
\end{verbatim}
\end{frame}

%----------- slide --------------------------------------------------%
\begin{frame}[fragile]
\frametitle{Some variations on the theme}

We could operate on multiple input files

\begin{verbatim}
import sys
wrotehdr=False
for f in sys.argv[1:]:
  fp=open(f)
  hdr=fp.readline().strip()
  if not wrotehdr:
    print hdr
    wrotehdr=True
    for l in fp:
      if l[3] != 'human': continue
      flds=l.strip().split(',')
      flds[4]=flds[4][:-3]
      print ','.join(flds)
\end{verbatim}
\end{frame}



\end{document}
